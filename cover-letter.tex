\documentclass[11pt]{letter}
\usepackage[utf8]{inputenc}
\usepackage{geometry}
\usepackage{graphicx}
\usepackage{fancyhdr}
% Control the font size
\usepackage{anyfontsize}

% Set the color of hyperlinks
\usepackage[colorlinks=true]{hyperref}
\hypersetup{
    pdftitle={Cover Letter},
    pdfauthor={Leonardo Uieda},
    linkcolor=black,
    citecolor=black,
    filecolor=black,
    urlcolor=black
}

\longindentation=0pt

% Redefine the empty style so the front page also has a header.
\fancypagestyle{empty}{ %
    \fancyhf{} % remove everything
    \lhead{}
    %\cfoot{\thepage}
    \cfoot{}
    \renewcommand{\headrulewidth}{0pt}
    \renewcommand{\footrulewidth}{0pt}
    \setlength\headheight{31pt}
}
\pagestyle{empty}

\signature{ Leonardo Uieda }
\address{
    1680 East-West Road, POST 804
    \\
    Honolulu, HI, USA, 96822
    \\
    email: \href{mailto:leouieda@gmail.com}{leouieda@gmail.com}
    \\
    phone: +1 808 428 4521
}

\begin{document}

\begin{letter}{
    Department of Statistics, Faculty of Science
    \\
    3182 Earth Sciences Building, 2207 Main Mall
    \\
    Vancouver, BC, Canada, V6T1Z4
}
\opening{Dear Members of the Search Committee:}

I am writing to apply for the position of Lecturer in the Masters of Data
Science Program of the University of British Columbia, as advertised on the
website of the Department of Statistics.
I am currently a Visiting Research Scholar at the University of Hawai'i in the
Department of Geology and Geophysics.
My research interests include geophysical inverse problems, scientific
software development, and geospatial data visualization.

% Research
My research training was in the development of computationally efficient
algorithms for solving geophysical inverse problems.
The focus of my graduate research was in estimating subsurface density
variations from observation of disturbances in the Earth's gravity field (e.g.,
Uieda and Barbosa, 2012, 2017).
Recently, I have been investigating the application of machine learning and
data science techniques to geophysical datasets,
in particular to long-term GPS measurements
(Uieda and Wessel, 2018;
\href{https://github.com/leouieda/aogs2018-gps}{github.com/leouieda/aogs2018-gps}).
I intend to continue my research program in the intersection of geophysics and
data science and would welcome collaborations with the Data Science Lab.
I would also use this opportunity to establish a relationship the members of
the Department of Earth, Ocean and Atmospheric Sciences and the
UBC Geophysical Inversion Facility.

% Software
To support my research and teaching efforts, I have created the open-source
Python library \textit{Fatiando a Terra}
(\href{http://www.fatiando.org/}{www.fatiando.org}).
The project uses software development best practices, such as version control
and the Github pull request workflow,
automated tests and continuous integration,
and packaging and distribution through the Python Package Index and
conda-forge.
My current work at the University of Hawai'i is to develop a Python wrapper
library for the \textit{Generic Mapping Tools}
(GMT; \href{http://gmt.soest.hawaii.edu/}{gmt.soest.hawaii.edu}),
an open-source software packaged widely used across the Earth, Atmospheric, and
Ocean Sciences to process and visualize geospatial data.
I have also contributed to other open-source projects, all of which can be
accessed through my Github profile
(\href{https://github.com/leouieda/}{github.com/leouieda}).

% Open science
As a proponent of open and reproducible science, I publish all of the
source-code and data for my first author publications through my research
group's Github page
(\href{https://github.com/pinga-lab}{github.com/pinga-lab}).
To promote reproducible research best practices within the group,
I maintain a template
(\href{https://github.com/pinga-lab/paper-template}{github.com/pinga-lab/paper-template})
which is used for creating new research projects.
I am interested in exchanging information and learning from the experiences of
the Data Science Workflows course of the MDS program.

% Teaching

I have three years of teaching experience from my work as Assistant Professor
at the State University of Rio de Janeiro (UERJ), Brazil.
I had the opportunity to design two geophysics courses for the Geology
program
(\href{http://www.leouieda.com/teaching/geofisica1.html}{leouieda.com/teaching/geofisica1.html}
and
\href{http://www.leouieda.com/teaching/geofisica2.html}{leouieda.com/teaching/geofisica2.html})
and a programming and numerical methods course for the Oceanography
program
(\href{http://www.leouieda.com/teaching/matematica-especial.html}{leouieda.com/teaching/matematica-especial.html}).
All three are based on active learning principals and rely heavily on hands-on
exercises using Jupyter notebooks, numerical simulations, and real world
datasets.
Each module of the geophysics courses includes a Jupyter notebook with
interactive simulations or real datasets to guide the students through a
series of formative and summative assessments
(e.g.,
\href{https://github.com/leouieda/geofisica2/blob/master/notebooks/1-ondas-sismicas.ipynb}{github.com/leouieda/geofisica2/blob/master/notebooks/1-ondas-sismicas.ipynb}).
The programming course was entirely implemented using Github repositories and
Github Classroom (\href{https://classroom.github.com/}{classroom.github.com}).
Each module has a repository with a group project containing instructions, a
Jupyter notebook, and data
(\href{https://github.com/mat-esp}{github.com/mat-esp}).
The students submit their work as repositories in a separate Github
organization
(e.g., the projects for the
2016 class are at
\href{https://github.com/mat-esp-2016}{github.com/mat-esp-2016}).
Grading and feedback are provided through Github issues (e.g.,
\href{https://github.com/mat-esp/about/issues/259}{github.com/mat-esp/about/issues/259}).
This workflow allowed me to manage a project-based class with over 70 students
as the sole instructor.
I have also taught short workshops on Python programming and inverse problems
in geophysics.
All of my teaching material is available on Github and on my personal website
(\href{http://www.leouieda.com/teaching/}{leouieda.com/teaching}).
Evidence of my public speaking skills is also publicly available through my
talks at the Scipy Conference, which have been recorded and uploaded to YouTube
(see \href{http://www.leouieda.com/talks/}{leouieda.com/talks}).


% Conclusion
I look forward to the opportunity to learn from the experience of the MDS
faculty and to use my geoscience expertise to expand and enrich the program.
I am particularly interested in the Capstone projects, for which I see great
potential for collaboration with the mining and oil and gas industries.

\closing{Thank you for your consideration,}

\ps
\vspace{1cm}
P.S. Please find attached the contact information for three references.
\newpage
{\fontsize{16pt}{0}\selectfont References for Leonardo Uieda}
\\[1cm]
Dr. Valéria C. F. Barbosa
\\
Full Researcher, Departamento de Geofísica, Observatório Nacional
\\
Rua Geneneral José Cristino, 77
\\
Rio de Janeiro, RJ, Brazil, 20921-400
\\
phone: +55~21~35049235
\\
email: \href{mailto:valcris@on.br}{valcris@on.br}
\\[0.5cm]
Dr. Paul Wessel
\\
Department of Geology and Geophysics, SOEST, University of Hawai'i at M\={a}noa
\\
1680 East-West Rd., POST 806
\\
Honolulu, HI, USA, 96822
\\
phone: +1~808~956~4778
\\
email: \href{mailto:pwessel@hawaii.edu}{pwessel@hawaii.edu}
\\[0.5cm]
Dr. Matt Hall
\\
Founder, Agile Scientific
(\href{https://agilescientific.com}{www.agilescientific.com})
\\
PO Box 336
\\
Mahone Bay, NS, Canada, B0J 2E0
\\
phone: +1~902~980~0130
\\
email: \href{mailto:matt@agilescientific.com}{matt@agilescientific.com}

\end{letter}

\end{document}
