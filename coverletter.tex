\documentclass[11pt]{letter}
\usepackage[utf8]{inputenc}
\usepackage{geometry}
\usepackage{graphicx}
\usepackage{fancyhdr}

\usepackage[colorlinks=true]{hyperref}
\hypersetup{
    pdftitle={Cover Letter},
    pdfauthor={Leonardo Uieda},
    linkcolor=black,
    citecolor=black,
    filecolor=black,
    urlcolor=black
}

\longindentation=0pt

% Redefine the empty style so the front page also has a header.
\fancypagestyle{empty}{ %
    \fancyhf{} % remove everything
    \lhead{}
    %\cfoot{\thepage}
    \cfoot{}
    \renewcommand{\headrulewidth}{0pt}
    \renewcommand{\footrulewidth}{0pt}
    \setlength\headheight{31pt}
}
\pagestyle{empty}

\signature{ Leonardo Uieda }
\address{
    Department of Geology and Geophysics
    \\
    University of Hawai'i at M\={a}noa
    \\
    1680 East-West Road, POST 804
    \\
    Honolulu, HI, USA, 96822
    \\
    email: \href{mailto:leouieda@gmail.com}{leouieda@gmail.com}
    \\
    phone: +1 808 428 4521
}

\begin{document}

\begin{letter}{
    Department of Statistics
    \\
    Faculty of Science
    \\
    3182 Earth Sciences Building, 2207 Main Mall
    \\
    Vancouver, BC, Canada, V6T1Z4
}
\opening{Dear Members of the Search Committee:}

I am writing to apply for the position of Lecturer in the Masters of Data
Science Program of the University of British Columbia, as advertised on the
Department of Statistics website.
I am currently visiting research scholar at the University of Hawai'i in the
Department of Geology and Geophysics.
My research interests include geophysical inverse problems, scientific
software development, and geospatial data visualization.

% Research and OSS
My research training is in the development of methods and algorithms for
solving geophysical inverse problems.
The methods that I have created since my PhD are computationally efficient
solutions to the inverse problem of estimating subsurface density variations
from observation of disturbances in the Earth's gravity field (e.g., Uieda and
Barbosa, 2012, 2017).
Recently, I have been interested in the application of machine learning and
data science techniques to geophysical datasets
(e.g., Uieda and Barbosa, 2017, and Uieda and Wessel, 2018;
\href{https://github.com/leouieda/aogs2018-gps}{github.com/leouieda/aogs2018-gps}).

Simultaneously, I have created open-source software to support my research and
teaching in the form of the Python library \textit{Fatiando a Terra}
(\href{http://www.fatiando.org/}{www.fatiando.org}).
The project uses software development best practices, such as version control
and the Github pull request workflow
(\href{https://github.com/fatiando/fatiando}{github.com/fatiando/fatiando}),
automated tests and continuous integration,
and packaging and distribution through the Python Package Index and
conda-forge.
My current work at the University of Hawai'i is to develop of a Python
wrapper library for the \textit{Generic Mapping Tools} (GMT;
\href{http://gmt.soest.hawaii.edu/}{gmt.soest.hawaii.edu}),
which is widely used across the Earth, Atmospheric, and
Ocean Sciences to process and visualize geospatial data.
I have also contributed to other open-source projects, all of which can be
accessed through my Github profile
\href{https://github.com/leouieda/}{github.com/leouieda}.

As a proponent of open and reproducible science, I publish all of the
source-code and data for my first author publications through my research
group's Github page
(\href{https://github.com/pinga-lab}{github.com/pinga-lab}).
To help promote reproducible research best practices within the research group,
I maintain a template
(\href{https://github.com/pinga-lab/paper-template}{github.com/pinga-lab/paper-template})
which is used for creating new projects and experimenting with computational
workflows.

% Teaching

Before starting my current position, I worked for three years as assistant
professor at the State University of Rio de Janeiro (UERJ), Brazil.

There, I taught undergraduate-level classes on geophysical methods,
introduction to geology, and programming and numerical methods.

I had the opportunity to create the Geophysics I and II and the Programming
and Numerical Methods courses.

All rely heavily on Jupyter notebooks, numerical simulations, and real world
datasets.

My geophysics classes make use of Jupyter notebooks coupled with Fatiando a
Terra to provide the students with an interactive environment for exploring a
given topic.

Each class module includes a notebook with interactive simulations and/or real
data to help guide the students through a series of formative and summative
assessments.

My programming course is entirely implemented as Github repositories using
Github Classroom.
Each module has a repository with a group project.
The students submit their work as git repositories (e.g., the projects for the
2016 class are at https://github.com/mat-esp-2016) and the grading and feedback
are provided in Github issues (e.g.,
https://github.com/mat-esp/about/issues/259).
This workflow allowed me to manage a project-based class with over 70 students
as the sole instructor.

I have also taught short workshops on Python programming and geophysics.

All of my teaching material is available on Github with links and descriptions
on my personal website http://www.leouieda.com/teaching/.

Evidence of my public speaking skills is publicly available through my talks at
the Scipy Conference, which have been recorded and uploaded to YouTube (see
\href{http://www.leouieda.com/talks/}{leouieda.com/talks}).


% Conclusion

I am confident that my research and teaching experience make my qualified to
teach a range of classes from the UBC MDS program, including (but not limited
to) programming, data visualization, software development, regression, and
machine learning.

I am particularly interested in the Data Science Workflows class, a subject in
which I have been recently working
(http://www.leouieda.com/blog/paper-template.html).

Collaborate on the Capstone projects and possibly bring in problems from Oil
and Gas and mining industries.

I can bring the experience of a different field to the data science program.

I intend to continue my research program with external collaborators.

I would like to collaborate with colleagues from the Data Science Lab to solve
interesting problems in Geophysics and Data Science.

I would also like to connect the Department of Statistics with specialists in
geophysical inversion from the Earth Sciences like Oldenburg and Haber.


\closing{Thank you for your consideration,}

\end{letter}

\end{document}
